\documentclass[11pt,a4paper]{article}
\usepackage{geometry}
\geometry{a4paper,scale=0.86}

\usepackage{CJKutf8}%支持中文 
\usepackage{color}
\usepackage{amsmath}
\usepackage{amssymb}
\usepackage{graphicx}
\usepackage{overpic}
\usepackage{authblk}
\usepackage{epsfig}
\usepackage{epstopdf}
\usepackage{colortbl}
\usepackage{multirow}
\usepackage{listings} 
\definecolor{gongnvlan}{rgb}{0.15,0.39,0.46}
\definecolor{tuerqi}{rgb}{0.0,0.587,0.412}
\definecolor{juhong}{rgb}{0.80,0.20,0.00}
\definecolor{orgion}{rgb}{0.95,0.95,0.92}
\definecolor{gold}{rgb}{0.50,0.42,0.05}
\definecolor{qianlan}{rgb}{0.5, 0.3, 0.36}
\lstset{
    basicstyle = \footnotesize,
    backgroundcolor= \color{orgion},
    keywordstyle= \color{tuerqi},
    commentstyle= \color{juhong}, 
    frame=shadowbox,
    rulesepcolor= \color{gongnvlan},
    escapeinside=``,
    xleftmargin=4em,xrightmargin=1em, aboveskip=1em,
    %ekatwhitespace = false,        
    captionpos = b,    
    extendedchars = false,    
    framerule= 1.0pt,
    rulecolor=\color{qianlan},
    keepspaces=true,
    otherkeywords={string},
    numbers=left,
    numbersep= 6pt,
    numberstyle=\tiny\color{gold},
    showspaces=false,
    showstringspaces=false,
    showtabs=false,
    stepnumber=1,
    stringstyle=\color{red},        % string literal style
    tabsize=2,
    breaklines = true
}
\uchyph=0
\lefthyphenmin=1
\righthyphenmin=1

\title{MultiDalitz简介}
\author{X.X Ma}
\begin{document}

\begin{CJK*}{UTF8}{gbsn}
\maketitle
\section{前言}
这个软件包“MultiDalitz”专为$J/\psi \to e^{+} e^{-} K^{+} K^{-} \pi^{0}$
而写。“Multi”意为多,“Dalitz”意为dalitz分析,整体的意义是多个dalitz的联合分波分析,
这是由于$\eta(1405)$和$\eta(1475)$难分开,但是各自的动力学又迥然不同,必须同时对
它们的dalitz plot进行联合拟合。这个软件包目前放在GitHub上,可以任意获取使用。
\section{软件包的安装}
在shell的命令行,可以用git直接获取源程序
\begin{lstlisting}[language=bash] 
 git clone git@github.com:xxmawhu/MultiDalitz.git 
\end{lstlisting}
唯一的依赖是ROOT的环境,配置好ROOT环境后,可以直接编译后使用。
\begin{lstlisting}[language=bash] 
 cd MultiDalitz
 make
\end{lstlisting}
编译完成后会自动在文件夹lib下生成一个.lib结尾的链接文件,名字为libKsKsK.so,在ROOT里面直接调用这个lib文件就可以使用
这个包。具体的调用语法为
\begin{lstlisting}[language=c++] 
 gSystem->Load("/path/to/lib/libKsKsK.so");
\end{lstlisting}
需要提供到libKsKsK.so的完整路径。

\section{使用}
\subsection{准备工作}
需要提供三个样本,a)data文件,b)PHSP MC文件,c)未经事例挑选的truth信息文件。需要注意的是样本b,c需要来自同一批MC样本,
因为程序需要加权平均求效率。
文件的格式为.txt,每三行是一个事例,存储的模板为
\begin{lstlisting}[language=c++] 
 E1 px1 py1 pz1
 E2 px2 py2 pz2
 E3 px3 py3 pz3 weight
\end{lstlisting}
我们约定第一行表示第“1”个粒子的信息,记为A,依次类推。比如可以选择A、B、C分别为$K^{+}$、$K^{-}$、 $\pi^{0}$。weight为该是事例的权重,用来加权处理系统误差。
约定a、b、c的文件名分别是Data.dat, PHSP.dat, Frac.dat。

\subsection{声明}
类的声明为
\begin{lstlisting}[language=c++] 
 MultiDalitPdf ksksKpdf("ksksKpdf", "ksksKpdf",
            v11, v12, v13, v14, v21, v22, v23, v24, v31, v32, v33, v34,
            PHSPdat);
\end{lstlisting}
其中$v11$等变量为四动量信息。
\begin{lstlisting}[language=c++] 
  double high = 1.8865;
  double low = 0-high;
  const double pi = 3.1415926*4;
  RooRealVar v11("v11", "v11", low, high);
  RooRealVar v12("v12", "v12", low, high);
  RooRealVar v13("v13", "v13", low, high);
  RooRealVar v14("v14", "v14", low, high);
  RooRealVar v21("v21", "v21", low, high);
  RooRealVar v22("v22", "v22", low, high);
  RooRealVar v23("v23", "v23", low, high);
  RooRealVar v24("v24", "v24", low, high);
  RooRealVar v31("v31", "v31", low, high);
  RooRealVar v32("v32", "v32", low, high);
  RooRealVar v33("v33", "v33", low, high);
  RooRealVar v34("v34", "v34", low, high);
\end{lstlisting}
设置好Frac.dat
\begin{lstlisting}[language=c++] 
 ksksKpdf.setFracDat(Fracdat);
\end{lstlisting}
添加母粒子
\begin{lstlisting}[language=c++] 
 RooRealVar spin0("spin0", "spin0",  0);
 spin0.setConstant();
 RooRealVar rRes("rRes", "", 3);
 rRes.setConstant();
 RooRealVar meta1475("meta1475", "meta1475", 1475E-3);
 meta1475.setConstant();
 RooRealVar weta1475("weta1475", "weta1475", 90E-3, 10e-3, 200e-3);
 //weta1475.setConstant();
 ksksKpdf.configMother("eta(1475)", //the name of this resonance
     "eta(1475)", // the title
     LineShape::RBW, // the lineshape: you can choose from RBW, BW, GS, Flatte, Flat.
     RooArgList(spin0, rRes, meta1475, weta1475) // the configure number: spin, effective radius, mass, width 
     );
\end{lstlisting}
添加共振态
\begin{lstlisting}[language=c++]
 const int S_WAVE(0);
 ksksKpdf.addResonance("a0_980", 
                       "eta(1405)", // this resonance from "eta(1405)"
                        LineShape::a980_0, // the lineshape
                        DecayType::AB, // this resonance decay into A,B
                        Rhoa0_980, Phia0_980,  // magnitude and phase angle
                        RooArgList(spin0, rRes, ma0, gPiEtaa0 , gKK2a0), // configure
                        S_WAVE // the a_0 and pi0 form  a S-wave
                        );
\end{lstlisting}

读取Data.dat
\begin{lstlisting}[language=c++]
RooArgSet theSet1, theSet2, theSet3;
    theSet1.add(RooArgSet(v11, v12, v13, v14, v21, v22, v23, v24));
    theSet2.add(RooArgSet(v31, v32, v33, v34));
    theSet3.add(RooArgSet(weight));
    RooArgSet theSet4(theSet1, theSet2, "");
    RooArgSet theSet(theSet4, theSet3, "");

    RooDataSet *data = RooDataSet::read(Datadat, theSet);
    data->Print();

    RooDataSet *datWeight = new RooDataSet(data->GetName(), data->GetTitle(),
                data,*data->get(), 0, weight.GetName());
    datWeight->Print();
\end{lstlisting}
开始拟合
\begin{lstlisting}[language=c++]
 RooFitResult *result = ksksKpdf.fitTo(*datWeight, Save(kTRUE));
\end{lstlisting}
获取分支比,并得到拟合结果
\begin{lstlisting}[language=c++]
 ksksKpdf.fitFractions(cout);
 ksksKpdf.project("project.root");
\end{lstlisting}

\end{CJK*}

\end{document}
